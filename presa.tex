\documentclass{beamer}
\usepackage[utf8]{inputenc}

\usetheme{Madrid}
\usecolortheme{default}
%--------------------------------------
\usepackage[T2A]{fontenc}
\usepackage[utf8]{inputenc}
\usepackage[russian]{babel}
%--------------------------------------

\hypersetup{
    colorlinks=true,
    linkcolor=blue,
    filecolor= blue,      
    urlcolor= blue,
    pdftitle={Overleaf Example},
    pdfpagemode=FullScreen,
    }
\title[System for Satellite Imagery Processing using SVM]
{Inroducion to LaTex}

\author[Yakunin]
{I.~V.~Yakunin\inst{1}}

\institute[HSE]
{
    \inst{1}
    Faculty of Computer Science \\
    Hse University
}
\date[27 07 2022] % (optional)
{LaTeX Course, July 2022}

\logo{\includegraphics[height=1cm]{photos/logo.png}}
\AtBeginSection[]

\begin{document}
\frame{\titlepage}

\begin{frame}
\frametitle{Table of Contents}
\tableofcontents
\end{frame}

\section{SVM model}

\begin{frame}
\frametitle{SVM training}
\begin{enumerate}
\item Picture preproccessing
\item HOG features extraction
\item Training SVM model
\end{enumerate}
\end{frame}
\begin{frame}{Preprocessing}
\begin{tabular}{c | c c c}
    Original photo & Y spectrum & U spectrum & V spectrum \\ \hline
    \includegraphics[width=0.1\textwidth, height=15mm]{photos/yuv1.jpg} & \includegraphics[width=0.1\textwidth, height=15mm]{photos/yuv2.jpg} & \includegraphics[width=0.1\textwidth, height=15mm]{photos/yuv3.jpg} & \includegraphics[width=0.1\textwidth, height=15mm]{photos/yuv4.jpg} \\
\end{tabular} \\
The idea here is to increase contrast on picture by switching to other color space
\end{frame}

\begin{frame}{HOG features extraction}
    \begin{tabular}{| r c | c |}
	\multicolumn{1}{r}{} &  \multicolumn{1}{c}{YUV} & \multicolumn{1}{c}{HOG} \\ \hline
	Y spectrum & \includegraphics[width=0.1\textwidth,height=15mm]{photos/yuv2.jpg} & \includegraphics[width=0.1\textwidth, height=15mm]{photos/hog1.jpg} \\ \hline
	U spectrum & \includegraphics[width=0.1\textwidth,height=15mm]{photos/yuv3.jpg} & \includegraphics[width=0.1\textwidth, height=15mm]{photos/hog2.jpg} \\ \hline
	V spectrum & \includegraphics[width=0.1\textwidth,height=15mm]{photos/yuv4.jpg} & \includegraphics[width=0.1\textwidth, height=15mm]{photos/hog3.jpg} \\ \hline
\end{tabular} \\
HOG algorithms extract features depending on the contrast between pixуls or blocks of pixels.
\end{frame}
\begin{frame}{SVM Model training}
    The SVM model is to be trained using HOG features which are labeled. It is supposed that after a training process the SVM model can distinguish pictures with ship from pictures without.
\end{frame}

\section {Pictures processing}
\begin{frame}{Sliding window technique}
Window Sliding Technique is a computational technique which aims to reduce the use of nested loop and replace it with a single loop, thereby reducing the time complexity. \\
\includegraphics[width=0.4\textwidth, height=60mm]{photos/window.png}
\end{frame}

\begin{frame}{Heatmap}
    New image size of the original is created with all pixels’ value equal to 0. Then the areas where a boat is detected are increased in value, whereas areas where a boat is not detected are decreased in value. \\
    \includegraphics[width=0.4\textwidth, height=60mm]{photos/heatmap.png}

\end{frame}
\end{document}
